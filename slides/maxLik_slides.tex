\documentclass{beamer}
%\documentclass[notes=show,handout]{beamer}
% \documentclass[handout]{beamer}

\mode<handout>{
\usepackage{pgfpages}
\pgfpagesuselayout{4 on 1}[a4paper,landscape,border shrink=3mm]
\setbeamercolor{background canvas}{bg=black!5}
}

\usepackage[latin1]{inputenc}
\usepackage{textcomp}
\usepackage{setspace}
\usepackage{bm}
\usepackage{Sweave}

\usetheme[right]{CopenhagenFOI}
\usecolortheme{sidebartab}
% \usepackage{beamerthemesplit}
% \setbeamertemplate{itemize subitem}{$\Rightarrow$}
\setbeamertemplate{enumerate item}[square]

\setbeamercolor{alerted text}{fg=red}

\usepackage{natbib}

\AtBeginDocument{
\setlength{\abovedisplayskip}{0.1em}
\setlength{\belowdisplayskip}{0.1em}
}
\makeatletter
\setbeamertemplate{note page}[plain]
\setbeamerfont{note page}{size=\scriptsize}
\hyphenpenalty = 5000
\binoppenalty = 10000
\relpenalty = 5000
\newcommand{\pkg}[1]{\textbf{#1}}
\newcommand{\code}[1]{{\color{blue}\texttt{#1}}}
\newcommand{\ELA}{\mbox{\large$\varepsilon$}}
\newcommand{\myhrule}{\vspace{1ex} \hrule \vspace{1ex}}
\newcommand{\argmin}{\mathop{\rm argmin}}
%\setbeamersize{text margin left=0.5cm}
\newcommand{\R}{\textsf{R}}
\newcommand{\noH}{H\hspace*{-0.67em}/\hspace*{0.15em}}
\newcommand{\noG}{G\hspace*{-0.6em}/\hspace*{0.1em}}
\makeatother


\title[maxLik: Maximum Likelihood in \R]{
   maxLik: A Package for Maximum Likelihood Estimation in \textsf{R}}
\author[Arne Henningsen]{Ott Toomet$^{1,2}$ and \underline{Arne Henningsen}$^3$}
\date{\footnotesize{
1. Department of Economics, University of Tartu\\
2. Department of Economics, Aarhus School of Business, University of Aarhus\\
3. Institute of Food and Resource Economics, University of Copenhagen}}

\begin{document}
\begin{frame}[plain]
\setlength{\textwidth}{1.15\textwidth}
\titlepage
\end{frame}

\section{Introduction}
\begin{frame}
\frametitle{Introduction}
Maximum Likelihood Estimation
\begin{itemize}
\item important technique in statistics and econometrics
\item software packages: ready-made routines for standard models
\item non-standard or new models: users have to implement them theirselves
   $\Rightarrow$ \pkg{maxLik} package could help
\item packages for MLE of specific models could use \pkg{maxLik}
   (\textbf{LambertW}, \textbf{mlogit}, \textbf{sampleSelection},
   \textbf{truncreg})
\item \pkg{maxLik} available on CRAN, R-Forge, and \url{maxLik.org}
\end{itemize}
\end{frame}

\section{Using maxLik}
\begin{frame}
\frametitle{Using \pkg{maxLik}}
ML estimation with the \pkg{maxLik} package
\begin{itemize}
\item function \code{maxLik}
\item 2 mandatory arguments:
   \begin{itemize}
   \item \code{logLik} = log-likelihood function
   \item \code{start} = vector of starting values
   \end{itemize}
\end{itemize}
\vspace*{2ex}
Example
\begin{itemize}
\item fit normal distribution
\item 100 draws, $\mu = 1$, $\sigma = 2$\\
   \code{x <- rnorm( 100, mean = 1, sd = 2 )}
\item ML estimation of $\mu$ and $\sigma$
\end{itemize}
\end{frame}

\begin{frame}
\frametitle{Using \pkg{maxLik}: Estimation}
Log-likelihood function (argument \code{logLik})
\begin{itemize}
\item first argument = vector of parameters
\item return: log-likelihood value (scalar) OR\\
   vector of log-likelihood values of each observation
\item \code{logLikFun <- function( param ) ...}
\end{itemize}
\vspace*{2ex}
Maximum likelihood estimation
\begin{itemize}
\item \code{mle <- maxLik( logLik = logLikFun,\\
   ~~~start = c( mu = 0, sigma = 1 ) )}
\item convergence in 7 iterations
\item estimated $\mu$ = sample mean
\item estimated $\sigma$ = sample std.\ dev.\ (no df cor.)
\end{itemize}
\end{frame}

\subsection{Gradients}
\begin{frame}
\frametitle{Using \pkg{maxLik}: Gradients}
Numerical and analytical gradients
\begin{itemize}
\item numerical calculation of gradients
   (\code{numericGradient})
\item optional argument \code{grad}:
   function to calculate (analytical) gradients
\item first argument = vector of parameters
\item return: vector of gradients OR\\
   matrix of gradients evaluated at each observation
\item \code{logLikGrad <- function( param ) ...}
\end{itemize}
\vspace*{2ex}
ML estimation with gradients
\begin{itemize}
\item \code{mleGrad <- maxLik( logLik = logLikFun,\\
   ~~~grad = logLikGrad,\\
   ~~~start = c( mu = 0, sigma = 1 ) )}
\item same estimates as before
\item standard errors slightly different
\end{itemize}
\end{frame}

\subsection{Hessian}
\begin{frame}
\frametitle{Using \pkg{maxLik}: Hessian}
Numerical and analytical Hessian
\begin{itemize}
\item numerical calculation of Hessian
   (\code{numericHessian} or \code{numericNHessian})
\item optional argument \code{hess}:
   function to calculate (analytical) Hessian
\item first argument = vector of parameters
\item return: Hessian matrix
\item \code{logLikHess <- function( param ) ...}
\end{itemize}
\vspace*{2ex}
ML estimation with gradients and Hessian
\begin{itemize}
\item \code{mleHess <- maxLik( logLik = logLikFun,\\
   ~~~grad = logLikGrad, hess = logLikHess,\\
   ~~~start = c( mu = 0, sigma = 1 ) )}
\item same estimates and standard errors as before
\end{itemize}
\end{frame}

\subsection{Algorithms}
\begin{frame}
\frametitle{Using \pkg{maxLik}: Optimisation algorithm}
\begin{itemize}
\item optional argument \code{method}
\item 5 optimisation methods
   \begin{itemize}
   \item \code{"NR"}: Newton-Raphson (default, G, H)
   \item \code{"BHHH"}: Berndt-Hall-Hall-Hausman (Gi, \noH{})
   \item \code{"BFGS"}: Broyden-Fletcher-Goldfarb-Shanno (G, \noH{})
   \item \code{"NM"}: Nelder-Mead (\noG{}, \noH{})
   \item \code{"SANN"}: simulated-annealing (\noG{}, \noH{})
   \end{itemize}
\item unused gradients (\noG) and Hessians (\noH) are ignored
\item NR: implemented in \pkg{maxLik} (\code{maxNR})
\item BHHH: uses the NR algorithm (wrapper: \code{maxBHHH})
\item BFGS, NM, SANN: use \code{optim}
   (wrapper: \code{maxBFGS}, \code{maxNM}, \code{maxSANN})
\end{itemize}
\end{frame}

\subsection{S3 class}
\begin{frame}
\frametitle{S3 class and methods}
Class
\begin{itemize}
\item \code{maxLik} returns an object of class \code{"maxLik"}
\end{itemize}
\vspace*{2ex}
Methods for Objects of Class \code{"maxLik"}
\begin{itemize}
\item \code{summary}
   \begin{itemize}
   \item calculates the covariance matrix of the estimates\\
      (inverse of the negative Hessian)
   \item calculates the standard errors, t-values, and P-values of the estimates
   \end{itemize}
\item print (also for summary.maxLik)
\item coef, vcov
\item logLik, AIC
\end{itemize}
\end{frame}


\section{Future}
\begin{frame}
\frametitle{Future Plans}
\begin{itemize}
\item Further optimization tools, e.g.
   \begin{itemize}
   \item \code{nlm} (G, H)
   \item \code{nlminb}: PORT routines, box constraints (G, H)
   \item L-BFGS-B (\code{optim}): box constraints (G)
   \item \code{constrOptim}: linear inequality constraints (G)
   \end{itemize}
\item Gradients and Hessian: ``attributes'' of objective function\\
   (next slide)
\item Features requested by users
\item Features suggested by you
\end{itemize}
\end{frame}

\begin{frame}
\frametitle{Future Issues}
Gradients and Hessian: ``attributes'' of objective function
\begin{itemize}
\item intermediate results for calculating the log-likelihood\\
   $\rightarrow$ calculation of gradients and Hessian\\
   $\Rightarrow$ less code and faster
\item user can choose: \code{grad}/\code{hess} OR attributes?
\item attributes could be implemented for NR, BHHH, (\code{nlm})
\item \code{nlm} uses attributes
   (compatibility: wrapper function calls \code{logLik}, \code{grad},
   and \code{hess} and returns LLV + attributes)
\item attributes could not be implemented for BFGS, (NM, SANN,)
   (L-BFGS-B, \code{constrOptim}, \code{nlminb})\\
   (\code{grad} and \code{hess} functions
   extract attribute \code{"gradient"} and \code{"hessian"}
   of \code{logLik})
\end{itemize}
\end{frame}


\section{Summary}
\begin{frame}
\frametitle{Summary}
\textbf{maxLik} package
\begin{itemize}
\item Simplifies ML estimations\\
   (users and package developers)
\item Gradients and Hessians: numerical or analytical
\item 5 different optimisation algorithms
\item Algorithm can be switched easily
\item Future plans
\end{itemize}
\end{frame}

\end{document}
