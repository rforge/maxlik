% -*- coding: utf-8 -*-
\documentclass[12pt,parskip=half]{scrartcl}
\usepackage[T1]{fontenc}
\usepackage[bookmarks=TRUE,
            colorlinks,
            pdfpagemode=none,
            pdfstartview=FitH,
            citecolor=black,
            filecolor=black,
            linkcolor=black,
            urlcolor=black,
            ]{hyperref}
\usepackage[utf8]{inputenc}
\usepackage{lmodern}
\usepackage{natbib}
\usepackage{icomma,xspace}

\begin{document}

Dear Friedrich, Peter and Søren:

\bigskip

Thank you for the comments we received for our paper ``maxLik: A
Package for Maximum Likelihood Estimation in R''.  We feel that the
referees have done an excellent job.  Below is the list of changes we
have introduced to the paper.

\subsection*{Reviewer 1}

\begin{itemize}
\item We mention functions \texttt{mle} and \texttt{mle2}, and discuss
  the advantages/novelty of \texttt{maxLik}.
\item We briefly describe the idea of sequential unconstrained
  maximization technique (SUMT).
\item We mention that the other ML-related packages do not provide
  \texttt{stdEr} or a direct analogue.
\item We streamlined the introduction of the log-likelihood function
   for the normal distribution.
\item As the \texttt{with}-construct cannot replace indexing
  everywhere (for instance, in case of vector- and unnamed
  parameter components), we mention this only in a footnote.
\item We re-worded the return messages, removing the ``maybe'' part.
  The return codes are not compatible with \texttt{optim()}
  but with \texttt{nlm()}.
\item We now refer to the numerically calculated gradients as
  ``finite-difference'' ones.
\item We added an extra argument \texttt{fixed} to
  \texttt{maxNR}, which allows to specify fixed parameters.  This
  addresses the reviewer's criticism.
  Furthermore, we added this argument also to \texttt{maxBFGS},
  \texttt{maxNM}, and \texttt{maxSANN}
  so that parameters can be fixed in all optimization methods.
  We have adjusted this paper accordingly.
\item We agree with the reviewer that automatic transforming parameters
  to fixed constants (section~3.3.2) may be a little complicated.  We
  also agree that the problems are easy to diagnose, and there are
  better algorithms available.  However, as this feature \emph{is}
  part of the \textbf{maxLik} package since its early days, and since for different
  reasons
  the user may choose not to use better-suited specific algorithms, we
  feel that section~3.3.2 may still have enough value to justify its
  publication.  However, if you do not agree with us, this section can be
  removed.
\item We have implemented all minor suggestions made by this reviewer.
\end{itemize}

\subsection*{Reviewer 2}

\begin{itemize}
\item Overlooked packages: see above.
\item We now mention the need for independent observations as a way to
  approximate the Hessian.
  As this paper does not intend to describe the details of the BHHH algorithm
  and information matrix equality, we refer to \citet[p.~490]{greene08} now.
\item In section 3.3.2 we clarify that our example works if replacing
  a general model with a more specific submodel by holding some parameters fixed.
  Hence, general constraints are not touched in this example.
\item We chose to retain the function \texttt{stdEr}, as it is a
  compromise between being easy to remember, short, following the R
  case traditions; and avoids breaking the API.
  The presence of a \texttt{vcov} method is mentioned in the last paragraph on
  page 3.
\item In case of presenting the gradient (and Hessian) of our sample
  likelihood function, we have chosen expositional simplicity over
  (slightly) more efficient code.
\item We have modified our paper according to all further suggestions
  made by this reviewer.
\end{itemize}

\bibliographystyle{spbasic}
\bibliography{references}

\end{document}
