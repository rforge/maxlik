% -*- coding: utf-8 -*-
\documentclass[a4paper]{article}
\usepackage[T1]{fontenc}
\usepackage[bookmarks=TRUE,
            colorlinks,
            pdfpagemode=none,
            pdfstartview=FitH,
            citecolor=black,
            filecolor=black,
            linkcolor=black,
            urlcolor=black,
            ]{hyperref}
\usepackage[utf8]{inputenc}
\usepackage{natbib}
\usepackage{icomma,xspace}

\begin{document}

Dear Editor,

thank you for the comments we received for our paper ``maxLik: A
Package for Maximum Likelihood Estimation in R''.  We feel that the
referees have done an excellent job.  Below is the list of changes we
have introduced to the paper.

\subsection*{Reviewer 1}

\begin{itemize}
\item We mention functions \texttt{mle} and \texttt{mle2}, and discuss
  the advantages/novelty of \texttt{maxLik}.
\item We briefly describe the idea of sequantial unconstrained
  maximization technique (SUMT).
\item We mention that the other ML-related packages do not provide
  \texttt{stdEr} or a direct analogue.
\item We streamlined the introduction of normal log-likelihood function.
\item Unfortunately, \texttt{with}-construct cannot replace indexing
  everywhere (for instance, in case of vector and unnamed
  parameter components).  Hence we left it to a footnote.
\item We re-worded the return messages, removing the ``maybe'' part.

  The return codes are not compatible with \texttt{optim()},
  but with \texttt{nlm()}.
\item We now refer to the numerically calculated gradients as
  ``finite-difference'' ones.
\item We added an extra argument \texttt{fixedPar} to
  \texttt{maxNR()}, which allows to specify fixed parameters.  This
  addresses the reviewer's critizism.
\item We agree with the reviewer that automatic fixing of the
  parameter values (section 3.3.2), may be a little complicated.  We
  also agree, that the problems are easy to diagnose, and there are
  better algorithms available.  However, as this feature \emph{is}
  part of the maxLik since it's early days, and since for some reasons
  the user may choose not to use better-suited specific algorithms, we
  feel that the sections may still have enough value.
\end{itemize}

\subsection*{Reviewer 2}

\begin{itemize}
\item Overlooked packages: see above.
\item We now explain the need for independent observations as a way to
  replace expectation with average.
\item In section 3.3.2 we clarify that our example works if replacing
  a general model with more specific submodel.  Hence general
  constraints are not touched in this example.
\item We chose to retain the function \texttt{stdEr}, as it is a
  compromise between being easy to remember, short, following the R
  case traditions; and avoids breaking the API.

  Presence of \texttt{vcov} method is mentioned in the 3rd paragraph,
  page 3.
\item In case of presenting the gradient (and Hessian) of our sample
  likelihood function, we have chosen expositional simplicity over
  (slightly) more efficient code.
\end{itemize}

We have also corrected a large number of typos, pointed by the referees.

\bibliographystyle{econometrica}
\bibliography{majandus}

\end{document}
