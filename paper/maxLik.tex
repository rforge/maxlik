\documentclass[10pt]{article}

\usepackage{url}

\renewcommand{\title}[1]{\begin{center}{\bf \LARGE #1}\end{center}}
\newcommand{\affiliations}{\footnotesize}
\newcommand{\keywords}{\paragraph{Keywords:}}

\setlength{\oddsidemargin}{0cm} \setlength{\evensidemargin}{0cm}
\setlength{\textwidth}{16.5cm} \setlength{\topmargin}{-1cm}
\setlength{\textheight}{24.5cm}

\newcommand{\code}[1]{\texttt{#1}}
\setlength{\emergencystretch}{3em}

\begin{document}
\pagestyle{empty}

\title{maxLik: A Package for Maximum\\[3mm]Likelihood Estimation in \textsf{R}}

\begin{center}
  {\bf Ott Toomet$^{1,2}$ and Arne Henningsen$^{3,4,*}$}
\end{center}

\begin{affiliations}
1. Department of Economics, University of Tartu (Estonia)\par
2. Department of Economics, Aarhus School of Business, University of Aarhus (Denmark)\par
3. Department of Agricultural Economics, University of Kiel (Germany)\par
4. Institute of Food and Resource Economics, University of Copenhagen (Denmark)\par
* Contact author: arne.henningsen@gmail.com
\end{affiliations}

\keywords Maximum Likelihood, Optimisation

\vskip 0.8cm

The \textbf{maxLik} package %~\cite{r-maxlik-0.5}
provides convenient tools
for maximum likelihood (ML) estimations
in the statistical software environment \textsf{R}.
This package is available from
CRAN (\url{http://cran.r-project.org/package=maxLik}),
R-Forge (\url{http://r-forge.r-project.org/projects/maxlik/}), and its
homepage (\url{http://www.maxLik.org/}).

The most important tool for a user of the \textbf{maxLik} package
is probably the \code{maxLik} function.
It is a wrapper function
that delegates the maximum likelihood estimation
to the selected optimisation routine.
Five optimisation methods are currently available
(names of the corresponding functions in parenthesis):
Newton-Raphson (\code{maxNR}),
Berndt-Hall-Hall-Hausman (\code{maxBHHH}), %\cite{berndt74},
Broyden-Fletcher-Goldfarb-Shanno~(\code{maxBFGS}),
Nelder-Mead~(\code{maxNM}), and
simulated-annealing~(\code{maxSANN}).
While the actual optimisation in
\code{maxBFGS}, \code{maxNM}, and \code{maxSANN}
is done by \code{optim},
the Newton-Raphson algorithm is implemented
in the function \code{maxNR} itself.
The actual optimisation in \code{maxBHHH}
is done by \code{maxNR}.

The first argument of \code{maxLik} (\code{loglik})
is mandatory and specifies the log-likelihood function.
Its first argument must be the vector of the parameters to be estimated
and it must return either a single log-likelihood value or
a numeric vector where each component is the log-likelihood value
corresponding to an individual observations.
The second and third argument (\code{grad} and \code{hess})
are optional and can be used to specify functions
that return the gradients and the Hessian of the objective function,
respectively.
If these functions are not provided by the user,
numerical gradients and Hessians are calculated if necessary.
The fourth argument (\code{start}) is mandatory and
must be used to specify a vector of starting values.
Finally, the fifth argument (\code{method}) is optional and
can be used to select the maximisation routine.
It defaults to \code{"NR"}, but it can also be \code{"BHHH"},
\code{"BFGS"}, \code{"NM"}, or \code{"SANN"}.  The \code{maxLik}
wrapper capabilities are designed in a transparent way, so that the
user may easily swap the methods without need the change the
arguments.  The arguments not used by a particular optimization
method, such as \code{hess} for the Berndt-Hall-Hall-Hausman method,
are ignored.

\code{maxLik} returns a list of class \code{"maxLik"}.
Corresponding methods can handle the likelihood-specific properties
of the estimate including the fact
that inverse of the negative Hessian is the variance-covariance matrix
of the estimated parameters.
The most important methods for objects of class \code{"maxLik"} are:
\code{summary} for returning (and printing) summary results,
\code{coef} for extracting the estimated parameters,
\code{vcov} for calculating the variance covariance matrix
of the estimated parameters,
\code{logLik} for extracting the log likelihood value, and
\code{AIC} for calculating the Akaike information criterion.

\textbf{maxLik} is implemented using S3 classes.

Currently, the \textbf{maxLik} package is used for maximum likelihood
estimations in three packages that are available on CRAN:
\textbf{mlogit}, \textbf{sampleSelection},
and \textbf{truncreg}.
On the useR! conference,
we would like to demonstrate how to use the \textbf{maxLik} package
for maximum likelihood estimations in \textsf{R}.
Furthermore, we would like to highlight its advantages and features
so that the number of users and package writers
who benefit from the \textbf{maxLik} package increases ;-)

%\nocite{ref1,ref2}
%\bibliographystyle{amsplain}
%\bibliography{agrarpol}

\end{document}
