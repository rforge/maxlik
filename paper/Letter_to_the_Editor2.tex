% -*- coding: utf-8 -*-
\documentclass[12pt,parskip=half]{scrartcl}
\usepackage[T1]{fontenc}
\usepackage[bookmarks=TRUE,
            colorlinks,
            pdfpagemode=none,
            pdfstartview=FitH,
            citecolor=black,
            filecolor=black,
            linkcolor=black,
            urlcolor=black,
            ]{hyperref}
\usepackage[utf8]{inputenc}
\usepackage{lmodern}
\usepackage{amsmath}
\usepackage[english]{babel}
\usepackage{graphicx}
\usepackage{natbib}
\usepackage{icomma,xspace}
% \input{isomath}
\newcommand{\code}[1]{\texttt{#1}}
\newcommand{\pkg}[1]{\textbf{#1}}
\newcommand{\proglang}[1]{\textsf{#1}}


\begin{document}

Dear Editor,

\bigskip

Thank you for the further comments on our paper ``maxLik: A
Package for Maximum Likelihood Estimation in R''.
We are happy that you and the referees are generally satisfied
with the revised version of our paper.
We are grateful particularly to the first referee
who pointed out some (minor) weaknesses
that were still present in the revised paper.
Below we respond to the suggestions of the first reviewer
and list the changes which we have introduced in the paper.


\subsubsection*{Reviewer 1}

\begin{itemize}
\item The reviewer is somewhat concerned about the complex examples.
  We agree that the density function of the normal distribution is not trivial.
  However, the normal distribution is perhaps the most widely used and
  best understood example for statistical distributions and likelihood
  functions.
  Therefore, we do not think that selecting a different statistical model
  would improve the clarity of the message.

  However, we agree that section 3.3.2 (fixing the parameters
  automatically) may feel a little complex.  As the rest of the paper
  does not depend on the presence of this section, we are willing to
  remove it, if you believe that this would benefit the potential readers.
\item We agree that the current state of a number of partially
  overlapping packages may not be optimal.  However, it seems to be
  less related to our package and more to the ``bazaar'' way of free
  software development.  We have added a related footnote (page 2),
  but feel that discussing various software development models will be
  beyond the scope of this paper.
\item We have fixed the typos and stylistic suggestions.
\end{itemize}

\subsubsection*{Changes in the maxLik package}

\begin{itemize}
\item
As we have continued the development of the package between the
revisions, we have also incorporated a few related changes in the paper.  In particular,
we mention an alternative way of providing the analytic gradient
(i.e.\ as attribute of the log-likelihood value) and
the new \code{maxBFGSR} routine, which is a BFGS optimizer
written solely in \proglang{R} and implemented in the \pkg{maxLik} package.
\end{itemize}


\end{document}
