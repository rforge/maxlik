% -*- coding: utf-8 -*-
\documentclass[a4paper]{article}
\usepackage[T1]{fontenc}
\usepackage[bookmarks=TRUE,
            colorlinks,
            pdfpagemode=none,
            pdfstartview=FitH,
            citecolor=black,
            filecolor=black,
            linkcolor=black,
            urlcolor=black,
            ]{hyperref}
\usepackage[utf8]{inputenc}
\usepackage{amsmath}
\usepackage[estonian]{babel}
\usepackage{graphicx}
\usepackage{natbib}
\usepackage{icomma,xspace}
\input{isomath}

\begin{document}

Dear Editor,

hereby we provide our feedback after the second round of reviews.

\section*{Reviewer 1}

\begin{itemize}
\item The reviewer is a womewhat concerned about the complex examples.
  We find the criticism a little hard to understand.  First, all the
  examples are essentially different ways of estimating the parameters
  of normal distribution.  As this is perhaps the most widely used and
  best understoor example of statistical distributions and likelihood
  functions, we do not think selecting a different statistical model
  would improve the clarity of the message.

  However, we agree that the section 3.2.2 (fixing the parameters
  automatically) may feel a little complex.  As the rest of the paper
  does not depend on the presence of that section, we are willing to
  remove it, if you believe that would benefit the potential readers.
\item We agree that the current state of number of partially
  overlapping packages may not be optimal.  However, it seems to be
  less related to our package and more to the ``bazaar'' way of free
  software development.  We have added a related footnote (page 2),
  but feel that discussing various software development models will be
  beyond the scope of this paper.
\item We have fixed the typos and stylistic suggestions.
\end{itemize}

As we have continued the development of the package between the
revisions, we also incorporated a few related changes.  In particular,
we mention an alternative way of providing the analytic gradient, and
\emph{maxBFGSR} routine, a BFGS optimizer, implemented in R in the
package. 


\end{document}
